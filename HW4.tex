\documentclass{article}
\usepackage{amsmath}
\usepackage{amsthm}
\usepackage{mathtools}
\usepackage[makeroom]{cancel}
\usepackage{physics}
\usepackage{braket}
\usepackage{bbold}
\usepackage{tensor}
\usepackage{titling}
\usepackage{fancyhdr}
\usepackage{mdframed}
\usepackage{xparse}
\usepackage{xcolor}
\usepackage{geometry}
\usepackage{graphicx}
\usepackage{epstopdf}
\usepackage{hyperref}
\usepackage{listings}
\usepackage[inline]{enumitem}
\usepackage{wrapfig}
\usepackage{float}
\usepackage[cal=boondox]{mathalfa}

\geometry{lmargin=.75in, rmargin=.75in}

%% For sample text
\usepackage{lipsum}

%% Resource-intensive packages
%\usepackage{chemformula}
\usepackage{siunitx}
%\usepackage[backend=bibtex,style=phys]{biblatex}

%\setlength{\textwidth}{\paperwidth}
%\addtolength{\textwidth}{-2in}
%\calclayout

%%Layout environments and macros
\makeatletter
\newcounter{problem}
\newcounter{problempart}
\newcounter{problemstart}
\setcounter{problem}{0}
\setcounter{problemstart}{1}
\newcommand{\theheadertitle}{}
\newcommand{\theproblemtitle}{}
\newcommand{\theproblemsubtitle}{}
\newcommand{\theproblemformattedtitle}{}
\newcommand{\theproblempartformattedname}{}
\newcommand{\theproblempartname}{}
\newcommand{\theproblemheadertext}{}
\newcommand{\theproblempage}{\the\numexpr\thepage-\theproblemstart+1\relax}

\NewDocumentEnvironment{problem}{ o o }%
{
	\refstepcounter{problem}
	\setcounter{problemstart}{\thepage}
	\setcounter{problempart}{0}
	\IfNoValueTF{#1}%
		{\renewcommand{\theproblemtitle}{Problem \theproblem}}
		{\renewcommand{\theproblemtitle}{#1}}	
	\IfNoValueTF{#2}%
		{%
			\renewcommand{\theproblemsubtitle}{}
			\renewcommand{\theproblemformattedtitle}{\textbf{\theproblemtitle}}%
		}%
		{%
			\renewcommand{\theproblemsubtitle}{#2}
			\renewcommand{\theproblemformattedtitle}{\textbf{\theproblemtitle} (\theproblemsubtitle)}%
		}	
	\renewcommand{\theproblemheadertext}{\theproblemtitle \, -- \, \theproblempage}
	\protected@edef\@currentlabelname{\theproblemtitle}
	\noindent \hrule height .5pt width \textwidth
	\vskip .75em
	\begin{center}
		\large{\theproblemformattedtitle}
	\end{center}
	\noindent \hrule height .5pt width \textwidth
	\vskip .75em
}{	\newpage }

\NewDocumentEnvironment{problemstatement}{ O{Problem Statement} O{\itshape} O{lightgray} }%
{
	\begin{mdframed}[backgroundcolor=#3, frametitle={#1}]#2
}{	\end{mdframed}}

\NewDocumentEnvironment{problempart}{ o tp tb }%
{
	\refstepcounter{problempart}
	\IfNoValueTF{#1}%
		{\renewcommand{\theproblempartname}{\alph{\theproblempart}}	}%
		{\renewcommand{\theproblempartname}{#1}	}
	\IfBooleanTF{#2}
		{\renewcommand{\theproblempartformattedname}{\textbf{(\theproblempartname)}}}
		{\renewcommand{\theproblempartformattedname}{\textbf{\theproblempartname}}	}%
	\IfBooleanTF{#3}
		{\renewcommand{\theproblempartformattedname}{\fbox{(\theproblempartformattedname)}	}}%
		{ }
	\theproblempartformattedname
}{}
\providecommand{\problempartautorefname}{Part}

\NewDocumentEnvironment{problempartsimple}{ o }
{
	\refstepcounter{problempart}
	\IfNoValueTF{#1}%
		{\renewcommand{\theproblempartname}{(\alph{problempart})}	}%
		{\renewcommand{\theproblempartname}{#1}	}
	\renewcommand{\theproblempartformattedname}{\fbox{\large{\textbf{\theproblempartname}}}}
	\vspace{1em}
	\theproblempartformattedname \quad
}

\NewDocumentEnvironment{problempartsimple*}{ o }
{
	\refstepcounter{problempart}
	\IfNoValueTF{#1}%
		{\renewcommand{\theproblempartname}{(\alph{\theproblempart})}	}%
		{\renewcommand{\theproblempartname}{#1}	}
	\renewcommand{\theproblempartformattedname}{\large{\textbf{\theproblempartname}}}
	\theproblempartformattedname \quad
}
\makeatother

\newcommand{\headertitle}[1]{\renewcommand{\theheadertitle}{#1}}
\numberwithin{equation}{problem}

\pretitle{% add some rules
  \begin{center}
    \Large\bfseries
}
\posttitle{%
  \end{center}%
%  \noindent\vrule height 2.5pt width \textwidth
}
\preauthor{%
  \begin{center}
  \begin{tabular}[t]{@{}l@{}}%
}
\postauthor{%
    \end{tabular}
    \vskip -.5em
    \par
  \end{center}%
}
\predate{%
  \begin{center}
}
\postdate{%
  \end{center}%
}

\pagestyle{fancy}
\fancyhf{}
\fancyhf[lf]{\thedate{}}
\fancyhf[rf]{\thepage}
\fancyhf[lh]{\theheadertitle}
\fancyhf[ch]{\theproblemheadertext}
\fancyhf[rh]{\theauthor}
\setlength{\footskip}{30pt}
\renewcommand{\footrulewidth}{.4pt}
\renewcommand{\footrule}{\hrule height \footrulewidth width \textwidth}

%% Referencing and layout macros
\newcommand{\eqnref}[1]{(\ref{#1})}
\newcommand{\eqnremark}[1]{\textnormal{(#1)}}
\newcommand{\nn}{\nonumber \\}

%% Spacing / text style macros
\newcommand{\tallphantom}{\phantom{\left( \int \dfrac{p_i^2}{2m} \right)}}
\newcommand{\tx}[1]{\textnormal{#1}}
\newcommand{\lqq}[1]{\quad \textnormal{#1}}
\newcommand{\byeqnqq}[1]{\qquad \tx{(by } \eqnref{#1} \tx{)}}
\newcommand{\byeqn}[1]{\tx{(by } \eqnref{#1} \tx{)}}
\newcommand{\spacedimplies}{\Longrightarrow \quad}
\newcommand{\ds}{\displaystyle}
\newcommand{\eff}{\mathcal{f}}

%% Physics macros
\newcommand{\cvec}[1]{\begin{pmatrix}#1\end{pmatrix}}
\newcommand{\dprime}{\prime \prime}
\newcommand{\til}[1]{\widetilde{#1}}

%% Theorem environments
\newtheorem{theorem}{Theorem}[problem]
\providecommand{\theoremautorefname}{Theorem}
\newtheorem{proposition}{Proposition}[problem]
\providecommand{\propositionautorefname}{Proposition}
\newtheorem{corollary}{Corollary}[proposition]
\providecommand{\corollaryautorefname}{Corollary}
\theoremstyle{definition}
\newtheorem{definition}{Definition}[problem]
\providecommand{\definitionautorefname}{Definition}
\newtheorem{example}{Example}[problem]
\providecommand{\exampleautorefname}{Example}
\theoremstyle{remark}
\newtheorem{remark}{Remark}[problem]
\providecommand{\remarkautorefname}{Remark}
\newtheorem*{note}{Note}

\providecommand{\includegraphicsautorefname}{Figure}
\providecommand{\figureautorefname}{Figure}

\providecommand{\problemautorefname}{Problem}
\providecommand{\problempartsimpleautorefname}{Part}
%\providecommand{\problempartsimple*autorefname}{Part}


\begin{document}
\author{Michael Haynes}
\title{Lasers: Assignment 4}
\headertitle{HW 4}
\date{06/02/2020}
\titlepage
\maketitle

\newpage
\begin{problem}[Problem 1][Nagourney 3.1] \label{3.1}

\begin{problemstatement}
A cavity is excited by a $\SI{500}{nm}$ mode-matched laser and has resonances whose separation is $\SI{125}{MHz}$ and whose width is $\SI{2.5}{MHz}$.  Determine the following:
\begin{enumerate}[label=(\alph*)]
	\item The cavity length $\ell$;
	\item The cavity finesse $\mathcal{F}$;
	\item The $Q$-factor;
	\item The photon lifetime $t_s$.
\end{enumerate}

\end{problemstatement}

\begin{problempartsimple}[(a)]
Let $\eff$ denote the free spectral range.  If $\ell$ is the round-trip cavity length (as opposed to the physical separation between the mirrors), then $\ds \eff = c / \ell$, so
\begin{align*}
\ell &= \frac{c}{\eff} \\
&= \frac{\SI{3e8}{m/s}}{\SI{125}{MHz}} \\
&\approx \SI{2.4}{m} \tx{.}
\end{align*}
\end{problempartsimple}

\begin{problempartsimple}[(b)]
The finesse is given by
\[
\mathcal{F} = \frac{\eff}{\Delta \nu_{1/2}} = \frac{\SI{125}{MHz}}{\SI{2.5}{MHz}} \approx 50 \tx{.}
\]
\end{problempartsimple}

\begin{problempartsimple}[(c)]
By definition,
\[
Q = \frac{\nu}{\Delta \nu_{1/2}} = \frac{(\SI{3e8}{m/s}) / (\SI{500}{nm})}{\SI{2.5}{MHz}} \approx \num{2.4e8} \tx{.}
\]
\end{problempartsimple}

\begin{problempartsimple}[(d)]
The lifetime is 
\[
t_c = \frac{1}{2\pi \Delta \nu_{1/2}} = \frac{1}{2\pi(\SI{2.5}{MHz})} \approx \SI{64}{ns} \tx{.}
\]

\end{problempartsimple}

\end{problem}

\begin{problem}[Problem 2][Nagourney 3.2] \label{3.2}

\begin{problemstatement}
Calculate the transmission of the electric field and intensity for a cavity having mirrors with field reflectivity (resp. transmissivity) $r_1, r_2$ ($t_1, t_2$), separation $d$, and containing a medium with transmission $t$.
\end{problemstatement}

As in the text, let $r_m \coloneqq r_2 t^2$, and let $\delta$ denote the round-trip phase shift through the cavity.  The ``one-pass'' contribution to the electric field just outside the output coupler (i.e., the field ignoring contributions from multiple passes through the cavity) is
\[
\vb{E}_1 = (t_1 t t_2) e^{-i \delta / 2} \vb{E}_0 \tx{.}
\]
Contributions from subsequent passes acquire factors of $r_1 r_2 t^2 e^{-i \delta} = r_1 r_m e^{-i \delta}$.  Let
\begin{gather*}
\alpha \coloneqq t_1 t t_2 e^{-i \delta / 2} \\
\beta \coloneqq r_1 r_m e^{-i \delta} \tx{.}
\end{gather*}
By the above logic, the total field at the output coupler is then
\begin{align*}
\vb{E}_t &= \alpha (1 + \beta + \beta^2 + \cdots) \vb{E}_0 \\
&= \alpha \qty( \frac{1}{1 - \beta} ) \vb{E}_0 \tx{.}
\end{align*}
The output intensity is thus
\begin{align*}
I_t &= I_0 \qty| \frac{\alpha}{1 - \beta} |^2 \\
&= I_0 \qty[ \frac{t^2 t_1^2 t_2^2}{(1 - r_1 r_m)^2 + 4 r_1 r_m \sin^2 (\delta / 2)} ]
\end{align*}
(the denominator is the same as for the reflected and circulating intensities).

\end{problem}

\begin{problem}[Problem 3][Nagourney 3.3] \label{ring_cavity}

\begin{problemstatement}
Calculate the FSR, finesse, $Q$, and photon lifetime for a three-mirror ring cavity with $\SI{20}{cm}$ arms and field reflectivities of $r_1 = r_2 = 0.99$, $r_3= 0.98$.  Assume $\lambda = \SI{600}{nm}$.
\end{problemstatement}

The round-trip optical path length is $\SI{60}{cm}$; hence the FSR is
\[
\eff = \frac{c}{\ell} = \frac{\SI{3e8}{m/s}}{\SI{60}{cm}} \approx \SI{500}{MHz} \tx{.}
\]
The finesse is approximately given by equation 3.14 from Nagourney:
\begin{align*}
\mathcal{F} &\approx \frac{\pi \sqrt{r_1 r_m}}{1 - r_1 r_m} \\
&= \frac{\pi \sqrt{r_1 r_2 r_3}}{1 - r_1 r_2 r_3} \\
&\approx 78 \tx{.}
\end{align*}
The $Q$-factor at $\lambda = \SI{600}{nm}$ is 
\[
Q = \mathcal{F} \qty( \frac{\ell}{\lambda} ) \approx 78 \qty( \frac{\SI{60}{cm}}{\SI{600}{nm}} ) \approx \num{7.8e7} \tx{.}
\]
Finally, the photon lifetime is 
\[
t_c = \frac{1}{2\pi \Delta \nu_{1/2}} = \frac{\mathcal{F}}{2\pi \eff} \approx \SI{2.5}{ns} \tx{.}
\]

\end{problem}


\end{document}